\chapter{Sauces}
\minitoc 

\recipe[This maginficent sauce is excellent with roast meats and game as served at the Ritz]{Bordelaise (Ritz London)}
%\serves{4}
\preptime{15 minutes}
\cooktime{1 hour}
\label{sauces:bordelaise}

\begin{ingreds}
	200g veal or beef trimmings
     	120g shallots or small onions
     	60g butter
     	70g button mushrooms, halved
    	2 streaky bacon rashers, chopped
	750ml (1 bottle) red wine
	50ml ruby port
	1 bay leaf
	1 thyme sprig
	700ml chicken stock (see~\ref{rec:chicken-stock})
	salt flakes
	pepper
%\columnbreak
%\ingredients[For the Crumble Mixture:]
%     80g Wholemeal Flour
%     80g Plain Flour
%     80g Butter (diced)
%     70g Demerara Sugar
\end{ingreds}

\begin{method}
     	Melt half the butter in a large saucepan over a medium--low heat.  When it is foaming, lay the trimmings in the pan.  Caramelize them gently for 15 minutes, turning them from time to time.

	Add the onions, mushroom, garlic and bacon.  Continue to cook until they are also cramelized; this will take about 5 minutes.

	Pour in the red wine, then the port, and add the bay leaf and thyme.

	On a steady boil, reduce by about $4/5^{th}$  until it is syrupy, concentrated and intense.

	Pour in the stock, bring to the boil, then reduce to a simmer.   Continue to cook at a simmer for 35-40 minutes, skimming regularly.

	When the sauce is ready it will be velvety and coat the back of a spoon.

	Correct seasoning with salt and pepper according to taste, then strain through a fine-mesh sieve.

	When ready to serve, reheat the sauce and mount it by dropping pieces of cold butter whilst whisking.  This will add texture and richness as well as rounding the flavours.
\end {method}

%\showit[1.25in]{example-image-b}{This is a picture}

\recipe[Traditional Italian Rag\'u. A lot of recipes are way too liquid, preventing the enjoyment of the meat flavours. The method described here ensures it has the right consistency.]{Rag\'u Bolognese}
\label{rec:bolognese}
\serves{6}
\preptime{20 minutes}
\cooktime{1 hour}

\begin{ingreds}
	250g minced beef
	250g minced pork
	2-3 rashers of high quality Pancetta, finely diced
     	2 medium onions, finely diced
	2 medium carrots, finely diced
	2 garlic cloves
     	25g butter
     	1 glass of dry Martini or Noilly Prat
	1 glass of chicken stock (see~\ref{rec:chicken-stock})
	1 1/2 tablespoon tomato pur\'ee
	salt
	pepper

\end{ingreds}

\begin{method}
	In a saut\'ee pan, coat the bottom with as little olive oil as possible and gently render the Pancetta cubes.

     	Whilst the Pancetta is cooking, melt  the butter in a medium saucepan over gentle heat.  When it is foaming, add the carrots and onions.  Caramelize them gently for 15 minutes, turning them from time to time.

	To better cook the minced meat, cut it perpendicular to it's strands and add both the pork and the beef to the pot.  Let it cook for 5-10 minutes at gentle heat with the lid closed.  Then increase the temperature and ensure all liquids have evaporated.  There is quite a bit of water in both; therefore ensure that there is no more collection of water at the bottom of the pan.  Stir frequenly as the end approaches in about 10-15 minutes at the higher heat.

	Add the vermouth and let it reduce and evaporate for 10 minutes.  No more water or liquid should be present at the bottom of the pan and the mixture should have a rather dry feel when stirred.

	Mix the tomato pur\'ee with the chicken stock and add to the mix, stirring until it has coated all the meat.

	Add the pancetta with all its fat and add the garlic.

	Keep stirring for a few minutes, then let the mixture rest for about 15 minutes, lid closed.
	
\end {method}