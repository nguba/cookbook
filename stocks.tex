\chapter{Stocks}
\minitoc

\section {Basic Principles} 

Escoffier once wrote: "\emph{Stock is everything in cooking.  Without it, nothing can be done}".  

Indeed, the quality of the stock is effectively the most important factor for a satisfactory result.  Therefore the cook mindful of success, will naturally direct his attention to the faultless preparation of his stock, exercising the most scrupulous care in their preparation.  

The importance of stocks in the kitchen is indicated by the French word for stock: fond, meaning “foundation” or “base.” In classical cuisine, the ability to prepare good stocks is the most basic of all skills because so much of the work of the entire kitchen depends on them. A good stock is the foundation of soups, sauces, and most braised foods and stews.

The following observations apply to all manufacture of stock:
\begin{itemize}
\item Never boil bones or the stock.  A gentle simmer at max \temp{97} is enough. Otherwise the end product will be cloudy.
\item Stocks are never salted since they are ingredients to other dishes.
\item Never boil a stock.  Bring to the boil and then reduce the temperature by adding cold water and skim immediately.  This technique is very efficient for the removal of scum that forms when cooking with meats such as beef.
\item Bones should be cooked for at least twelve hours to extract their minimum useful potential.
\item Optionally, vegetables may be roasted as well (use a shallow pan for this).
\item To avoid fermentation, rapid cooling of the stock is advisable.  This can be achieved by adding more cold water.  Immersing the finished stock in a waterbath and then rapid transfer to a fridge is also advisable.
\item A lot of the clarity of the stock depends on how well it has been skimmed and that its cooking has been conducted under gentle heat. 
\end{itemize}  



\recipe[This is a simple and effective method for obtaining a smooth and gelatinous stock, perfect for binding sauces.]{Chicken Stock \`a la Ritz}
\label{rec:chicken-stock}
%\serves{4}
\preptime{1 1/4 hour}
%\cooktime[Chill time]{11/2 hours}
\cooktime{6-8 hours}
%\vegetarian
%\freeze
\begin{ingreds}
	2kg chicken wings
     	2-3 tsp rapeseed or sunflower oil
     	20g butter, cut into cubes
     	1 onion, chopped
    	1 carrot, chopped
	1 leek, chopped
	1/2 head celery, chopped
	3 garlic cloves, chopped
	2.5--3.0 litres of water
%\columnbreak
%\ingredients[For the Crumble Mixture:]
%     80g Wholemeal Flour
%     80g Plain Flour
%     80g Butter (diced)
%     70g Demerara Sugar
\end{ingreds}

\begin{method}[Preheat the oven to Gas Mark 4, Electric \temp{180}, Fan \temp{160}.]
    Lay the chicken wings in a roasting tin, add a splash of oil and half of the butter.  Roast for about 1 hour or until golden.

    Place the onion, celery, garlic, carrot and leek into a seperate roasting tin or ovenproof dish, add a splash or oil and half the butter and roast until golden.

   	Drain the fat from the chicken wings and put with the vegetables in a large pot.  Cover with the water, bring to the boil and simmer gently for 6--8 hours.

	Pass the stock through a fine-mesh sieve, allow it to cool and then refrigerate in an airtight container until required.
\end {method}

\recipe[This forgotten recipe from the great Escoffier is one of my all time favourites.]{Escoffier's Fond Blanc Ordinaire}
\label{rec:escoffier-fonds-blanc}
%\serves{4}
\preptime{14 hours}
%\cooktime[Chill time]{11/2 hours}
\cooktime{6-8 hours}
\begin{ingreds}
	1.5kg beef bones (shin of beef)
	1.5kg lean beef
	750g fowls' skeletons
	450g carrots
	225g turnips
	340g leeks
	120g parsnips
	5.0 litres of water
\end{ingreds}

\begin{method}[Preheat the oven to Gas Mark 4, Electric \temp{180}, Fan \temp{160}.]
     	Break the bones as finely as possible, sprinkle with a bit of fat and brown them in an oven.  Stir them repeatedly.

	When the bones are lighlty browned, put them into as saucepan with five litres of cold water, add the vegetables and put to the boil.

	As soon as boil is reached, add a bit more cold water and skim carefully.  Then turn the heat to the lowest setting and smiller gently for twelve hours, lid half closed.

	Then roughly remove the fat (keep it for later), strain the liquid through a sieve and let it cool.

	Next put the meat in a saucepan just large enough to hold it, brown a little in some fat (ideally from the previous step), and brown it a little.  Then drain off fat entirely.

	Add about half a litre of the previously prepared stock, cover the saucepan and let the meat simmer until the stock is almost entirely reduced, turning it frequently.

	Now pour in the remainder of the previously prepared stock, bring to the boil and then continue very slowly, gently at a simmer under low heat, in order to cook the meat for about one hour and a half (keep the lid off). 

	 As soon as the meat is well cooked, remove the fat and pass the stock through a fine-mesh sieve, allow it to cool and then refrigerate in an airtight container until required.
\end {method}
