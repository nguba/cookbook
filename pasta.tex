\chapter{Pasta}
\minitoc

\recipe[Pasta dough with a lot of 'bite' due to the increased content in Semolina.]{Egg Pasta}
\label{rec:egg-pasta}
\preptime{1 1/2 hour}
\serves{4}

\begin{ingreds}
	100g Semolina
	300g Italian '00' flour
	4 large eggs
\end{ingreds}

\begin{method}[Eggs should not come directly from the fridge.  They should be at room temperature for best results.]		
     	Add the semolina and flour into a bowl.

	Break all the eggs over the bowl and mix with the round side of the spoon to ensure all flour is hydrated.  This is a rather dry dough, don't panic when it looks like there won't be enough water to hydrate all the flour.

	Continue mixing by hand, ensuring no flour is left over and shape the dough into a ball.

	Cover with wet towel and rest for 30-45 minutes to hydrate the dough.

	Knead the dough into a smooth paste.

	Cover with cling film and rest for one hour, alternatively leave in the fridge overnight.

\end {method}


\recipe{Spaghetti alle Vongole (Napoletana)}
\label{rec:pasta-spaghetti-vongole-napoli}
\preptime{min 5 hours}
\cooktime{1/2 hour}
\serves{4}

\begin{ingreds}
	400g Spaghetti
	1kg Clams (Vongole)
	3 Garlic Cloves
	150ml Olive Oil
	250g Pomodorini del Piennolo (optional)
	Fresh Parsley (according to taste)
	Sea Salt
\end{ingreds}

\begin{method}[Soak the clams at least for 5-6 hours in (salted cold water) to remove any remaining sand.]		
     	Add the semolina and flour into a bowl.

	Heat water for boiling the spaghetti.

	Add the olive oil to a pan, slice and add the garlic, let it brown until golden (light blonde) and remove from the pan when ready.

	Slice the pomodorini in half and add with the clams to the pan cotaining the garlic oil.

	As soon as the clams open, begin to cook the pasta.

	When ready, drain the juice of the clams into a fresh pan and add the cooked pasta and swirl to coat it with the juice.

	Plate and serve.

\end {method}




