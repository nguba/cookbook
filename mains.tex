\chapter{Main Courses}
\minitoc

\recipe[A parciluar favourite in Greece, where the distinct flavour of Rigani is particularly appreciated.  A favourite for BBQs]{Rigani Lamb or Chicken}
\preptime{15 minutes}
\cooktime{7-10 minutes}

\begin{ingreds}
	150ml olive oil
	juice of 1 lemon
	2 onions, peeled
	2 bay leaves cut into small pieces
	2 teaspoons of dried rigani
	pulp of 2 tomatoes
	salt and pepper
\end{ingreds}

\begin{method}[I often use a tin of peeled timatoes instead of pulp.  Marinading overnight is highly recommended.]
Add all the ingredients to a blender and mix in a couple of bursts.

Cube the meat and marinate for at least 3 hours, ideally overnight in the fridge.

Drain the cubes and thread them on to skewers.

Grill whilst basting regularly until the cubes are of a rich brown colour outside, but still pink and juicy within.
\end {method}

\recipe[]{Bavarian Roast Pork}
\preptime{15 minutes}
\cooktime{7-10 minutes}

\begin{ingreds}
	2kg shoulder or belly of Pork
	1 leek, sliced
	1 fat carrot, sliced
	1 reasonably large onion, chopped
	1-2 bottles of Bavarian Dunkel
	1 tsp paprika
	1 tsp caraway seeds
	salt and pepper according to taste
\end{ingreds}

\begin{method}[Pre-heat the oven to \temp{160}]
Whilst the oven is pre-heating, rub the meat with paprika and caraway seed.

Slightly cover a roasting pan with fat and lay the meat into it, skin side up.

Insert into the oven at mid-level and let it roast for about 10 minutes.

Take out the pan and insert the vegetables around the meat, adding enough beer so that the liquid is about 15cm (ca 1/2 inch) high.

Reintroduce to pan to the oven and gently roast for at least 1.5-2 hours according to taste.   Ensure the liquid never dries but remains moist within the vegetable mix.  This important for achieving a satisfactory sauce.

Roast pork is never served english (ie pink) but rather fully cooked.  The genltle slow-roasting prevents it from going tough.

Once the meat is done, remove the vegetables and the liquid in order to process the sauce by straining it through a suitable sieve (or chinois).

Ensure that the skin is suitably fattened again, and proceed to make the crackling by continuing to roast the joint skin side up at \temp{200}
\end {method}

\recipe[]{Fondue Savoyarde}
\preptime{15 minutes}
\cooktime{7-10 minutes}

\begin{ingreds}
	300g Comt\'e
	300g Guy\`ere
	300g AOP Beaufort
	750ml White Wine (Muscadet)
	2g Cornflour
	1 Garlic clove
	Grated nutmeg
	3 pinces Sea Salt
	Freshly ground black pepper
\end{ingreds}

\begin{method}[]
Remove the rind from the cheeses and cut each cheese into small cubes. 
Cut the bread into small pieces.

Rub the fondue pot with the garlic clove and then add the white wine 
(reserving a small glass). Place this over a low heat.

Add the corn starch to the small glass of white wine, mix together and 
add to the fondue pot. Add the cheese a handful at a time and stir until 
it melts. Stir in a figure of eight pattern until you have a smooth, 
creamy fondue. The fondue should resemble a thick sauce: if it is too thick, 
add a little more warmed white wine. If is too thin, add more cheese or corn 
starch mixed with white wine.

Remove the fondue pot from the heat and add the pepper, nutmeg and salt. Place 
the fondue pot on a hot plate or burner on the table to serve. Stir often to 
keep the fondue smooth.
\end {method}